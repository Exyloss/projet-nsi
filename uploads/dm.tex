\documentclass[12pt]{article}
\usepackage[french]{babel}
\usepackage[utf8]{inputenc}
\usepackage{fancyhdr}
\usepackage{geometry}
\usepackage{background}
\usepackage{xcolor}

enewcommand{
mdefault}{phv} % Arial

enewcommand{\sfdefault}{phv} % Arial
\geometry{
a4paper,
left=1.6in,
right=0.3in,
top=20mm,
}

\pagestyle{fancy}
ancyhf{}

head{Terminale C}
\lhead{Antonin PONS}


enewcommand{\maketitle} {
egin{center}
{\hugefseries
Étude de documents}
\end{center}
}

\SetBgScale{1}
\SetBgAngle{0}
\SetBgColor{black}
\SetBgContents{
ule{.4pt}{100in}}
\SetBgHshift{-9.5cm}

egin{document}
\maketitle

La IVème république succède au gouvernement provisoire de la république française 
instauré à la fin de la seconde guerre mondiale. Le but premier de ce régime consiste en 
la reconstruction du pays tant sur le volet économique que social.

Le document que nous allons analyser est une affiche scolaire publiée par les éditions 
Roussignol en 1953, durant la IVème république. Cette affiche a pour titre « Les 
Français se sont remis au travail ». Elle est destinée aux élèves de CM1 et a pour 
objectif de décrire la reconstruction de la France durant l’après-guerre. Lors de cette 
analyse de document, nous nous pencherons sur les innovations réalisées durant 
l’après-guerre exposées sur cette affiche. Puis, nous verrons en quoi cette affiche donne
aussi une image idéalisée de la reconstruction de la France après la Seconde Guerre
Mondiale.

space{0.5cm}
Tout d’abord, nous pouvons observer la présence d’un avion de ligne, en effet, le trafic 
aérien civil explosa durant cette période, le trafic aérien mondial passa de 9 millions 
de passagers en 1946 à 88 millions en 1958 grâce à l’arrivée des avions de ligne à 
réaction qui fit chuter le coût de la place au kilomètre.

De plus, cette affiche décrit aussi l’essor des autobus qui remplacèrent progressivement 
les tramways dans les grandes villes après la Seconde Guerre Mondiale. Cette guerre 
désorganisa totalement les transports parisiens, ce qui permis la naissance de la RATP.

Aussi, nous observons à droite de l’affiche la présence d’un village caractéristique du 
bassin minier du Nord-Pas-de-Calais derrière lequel on distingue des usines. En effet, 
l’effectif du Bassin minier passe de 91.500 en 1945 à 134.000 en 1946.
Ainsi, ce document met en lumière les progrès technologiques et l'augmentation 
de la production réalisés durant l’après-guerre, mais cette affiche ne relate pas une 
version exacte de la période de reconstruction.

space{0.5cm}
En effet, ce document ne décrit pas objectivementla reconstruction durant l’après-guerre
, les dégats causés par les différents bombardements en France sont relativisés. Par 
exemple, dans la ville de Brest, 4800 immeubles furent détruits sur un total de 11.700 
immeubles. Les baraques, des logements qui ne devaient être que provisoires, étaient 
encore habitées par 20.000 personnes en 1957 uniquement pour la ville de Brest.

D’autre part, le titre « Les Français se sont remis au travail » critique le régime de 
Vichy en sous-entendant que les Français travaillaient moins durant cette période. Mais 
la relance industrielle Française a été rendue possible grâce à la main d’oeuvre immigrée.
L'État met d'ailleurs en place l'Office national de l'immigration fin 1945 pour
contrôler les flux migratoires mais aussi pour les encourager, certains travailleurs
étant directement recrutés dans leurs pays d'origine.

space{0.5cm}
On en conclut que cette affiche permet de décrire efficacement les grands changements en 
France après la Seconde Guerre Mondiale pour des élèves de CM1, mais celle-ci idéalise 
aussi la reconstruction organisée par le nouveau gouvernement Français afin de maintenir 
la stabilité du nouveau régime.
\end{document}

